\section{ARNetworkAL protocol}

The \ARCode{ARNetworkAL} library is responsible for the network abstraction of the \ARCode{ARNetwork} library. Both libraries are really tied together, and the structure of a network packet is composed of header from both, but for the sake of clarity, this section will describe the entire \ARCode{ARNetworkAL+ARNetwork} headers.

Naming conventions: The full block of data sent by \ARCode{ARNetworkAL} is called a \ARCode{packet}. A \ARCode{packet} is composed of one or more \ARCode{frame}. Each \ARCode{frame} is composed of an \ARCode{header} (Described in this section), and some \ARCode{data} (Described in either \ARCode{ARCommand} or \ARCode{ARStream} section).

\subsection{Wifi}

On Wifi network, a \ARCode{packet} is sent on an UDP socket, and corresponds to an UDP packet. The destination ports were negotiated during the \ARCode{ARDiscovery.Connection}, and the sending ports are free.


A \ARCode{packet} can contain multiple \ARCode{frames}. \ARCode{frames} are simply added one after another in the UDP packet.


A \ARCode{frame} contains the following information:
\begin{itemize}
\item{Data type (1 byte)}
\item{Target buffer ID (1 byte)}
\item{Sequence number (1 byte)}
\item{Total size of the \ARCode{frame} (4 bytes, Little endian)}
\item{Actual data (N bytes)}
\end{itemize}

\subsubsection{Data types}

The \ARCode{ARNetworkAL} library supports 4 types of data:
\begin{itemize}
\item{Ack(1): Acknowledgment of previously received data} 
\item{Data(2): Normal data (no ack requested)}
\item{Low latency data(3): Treated as normal data on the network, but are given higher priority internally}
\item{Data with ack(4): Data requesting an ack. The receiver must send an ack for this data !}
\end{itemize}

The full list of data types can be found in the\\
\ARFile{libARNetworkAL}{Include/libARNetworkAL/ARNETWORKAL_Frame.h} file.


To acknowledge a data, simply send back a \ARCode{frame} with the Ack data type, a buffer ID of \ARCode{128+Data_Buffer_ID}, and the data sequence number as the data.\\
E.g. : To acknowledge the \ARCode{frame} \ARCode{"(hex) 04 0b 42 0b000000 12345678"}, you will need to send a \ARCode{frame} like \ARCode{"(hex) 01 8b 01 08000000 42"}

\subsubsection{Target Buffer IDs}

Buffers IDs are separated into three main sections:
\begin{itemize}
\item{[0; 9]: Reserved values for \ARCode{ARNetwork} internal use.}
\item{[10; 127]: Data buffers.}
\item{[128; 255]: Acknowledge buffers.}
\end{itemize}

By convention, Controller to Product buffers starts at 10 and grow up, while Product to Controller buffers starts at 127 and go down. A buffer number can be used in both direction.


A complete description of the buffers will be done in the \ARCode{ARNetwork} section.

\subsubsection{Sequence number}

Each buffer has its own independant sequence number, which should be increased on new data send, but not on retries. The \ARCode{ARNetwork} library will ignore out of order and duplicate data, but will still send Acks for them if requested. If the back-gap in sequence number is too high (we use 10 in \ARCode{ARNetwork} library), then the \ARCode{frame} is not considered out of order, and instead is accepted as the new reference sequence number.

\subsubsection{Multiple \ARCode{frames} in one \ARCode{packet}}

The \ARCode{packet} \ARCode{"(hex)01ba270800000042020bc30b00000012345678"} can be splitted in the follwing way:\
\begin{itemize}
\item{Read the type of the first \ARCode{frame}: \ARCode(0x01), or Ack}
\item{Read the buffer id of the first \ARCode{frame}: \ARCode(0xba)}
\item{Read the size of the first \ARCode{frame}: \ARCode{0x00000008} (written in human readable endian): 7 bytes of header + 1 byte of data}
\item{Read the data of the first \ARCode{frame}: \ARCode{0x42}}
\item{Since there is remaining data in the buffer, start again the process for a second frame}
\end{itemize}

For this exmample, we have an ack for buffer \ARCode{0x0a}, sequence number \ARCode{0x42}, and a non acknowelged data for buffer \ARCode{0x0b}, with content \ARCode{0x78563412}.


The \ARCode{ARNetworkAL/ARNetwork} libraries should never send ill-formed packets (i.e. every packet you receive consists of [1..N] \ARCode{frames}, without garbage data at the end. These libraries are ill-formed packets resistant: Before reading a \ARCode{frame}, we make sure that there is at least 7 bytes of data remaining in the buffer, before reading data, we make sure that there is at least \ARCode{data_size - 7} bytes of data in the buffer, and so on.

\subsubsection{Disconnection detection}

The \ARCode{ARNetworkAL} library will consider the remote to be disconnected if no data was read from the socket for the last 5 seconds.

\subsection{BLE}

On BLE networks, a \ARCode{packet} will only encapsulate a single \ARCode{frame}. A \ARCode{frame} is caracterized by the BLE characteristic address it belongs to.


A \ARCode{frame} contains the following information:
\begin{itemize}
\item{Data type (1 byte)}
\item{Sequence number (1 byte)}
\item{Actual data (up to 18 bytes)}
\end{itemize}

\subsubsection{Data types}

The \ARCode{ARNetworkAL} library supports 4 types of data:
\begin{itemize}
\item{Ack(1): Acknowledgment of previously received data} 
\item{Data(2): Normal data (no ack requested)}
\item{Low latency data(3): Treated as normal data on the network, but are given higher priority internally}
\item{Data with ack(4): Data requesting an ack. The receiver must send an ack for this data !}
\end{itemize}

The full list of data types can be found in the\\
\ARFile{libARNetworkAL}{Include/libARNetworkAL/ARNETWORKAL_Frame.h} file.


To acknowledge a data, simply send back a \ARCode{frame} with the Ack data type, to the characteristic of id \ARCode{16+characteristic_id}, and the data sequence number as the data.\\
E.g. : To acknowledge the \ARCode{frame} \ARCode{"(hex) 04 42 12345678"} in characteristic \ARCode{0xf00a}, you will need to send a \ARCode{frame} like \ARCode{"(hex) 01 01 42"} in characteristic \ARCode{0xf01a}.

\subsubsection{Characteristic IDs}


The products should declare 32 characteristics for \ARCode{ARNetwork}, plus 2 for ftp-like purpose (update, media download...).


The \ARCode{ARNetwork} characteristics are numbered from \ARCode{0xf000} to \ARCode{0xf01f}. They will be referred as 0 to 31 (just add \ARCode{0xf000} to get the real id).


Characteristics are separated into three sections:
\begin{itemize}
\item{[0; 9]: Reserved for \ARCode{ARNetwork} internal use.}
\item{[10; 15]: Data.}
\item{[16; 31]: Acknowledges.}
\end{itemize}

By convention, the Controller to Product characteristics start a 10 and grow up, while the Product to Controller characteristics start at 15 an go down. A characteristic can be used for two-way communication, but that is not advised (and not used on products).


The two characteristics for ftp-like transmission (using \ARCode{ARDataTransfer} library) are \ARCode{0xfd23} and \ARCode{0xfd53}.


\subsubsection{Sequence number}

Each characteristic has its own independant sequence number, which should be increased on new data send, but not on retries. The \ARCode{ARNetwork} library will ignore out of order and duplicate data, but will still send Acks for them if requested. If the back-gap in sequence number is too high (we use 10 in \ARCode{ARNetwork} library), then the \ARCode{frame} is not considered out of order, and instead is accepted as the new reference sequence number.

%%% Local Variables:
%%% mode: latex
%%% TeX-master: "main"
%%% End:
