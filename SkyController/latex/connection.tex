\section{Handling ARSDK connections}

\subsection{Connecting to a drone}

Connection to a drone is done in the same was as on any other Android device. Any app working on phones/tablet should work on the SkyController, as long as the UI is adapted to the DPAD navigation and the screen size.

\subsection{Connecting to a tablet/smartphone}

The actual code to handle incoming connections is part of the FreeFlight application embedded on SkyControllers. The code is not open-sourced, but it is implemented using components avaible in the ARSDK.

\subsubsection{Publishing a mDNS service}

To be discoverable on the network, the application should publish a mDNS service with the proper type (\ARCode{_arsdk-0903._udp.local}).

\subsubsection{Accepting incoming ARSDK connections}

To accept a connection, the application should have an instance of an \ARCode{ARDISCOVERY_Connection} object, running the \ARCode{ARDISCOVERY_Connection_DeviceListeningLoop()} method. This discover server will then negociate connection as would a drone do. A complete rundown of the \ARCode{ARDiscovery} protocol is available in the \href{http://developer.parrot.com//docs/bebop/ARSDK_Protocols.pdf}{ARSDK protocols documentation}.

\subsubsection{Creating the ARNetwork instances}

The ARNetwork instance can be created with the parameters from the Discovery phase. The instance is then used in the same way as a client instance (the ARNetwork protocol is peer to peer).

\subsubsection{Routing commands}

The minimum routing rules for the commands are the following:

\begin{itemize}
\item All \ARCode{ARCommands} received from the drone must be sent to the tablet (if present)
\item All \ARCode{ARCommands} received from the tablet must be sent to the drone (if present), except for the SkyController.X.Y commands (see \ARCode{ARCOMMANDS_Filter})
\end{itemize}

\emph{Note: In order to grab the \ARCode{ARCommands}, you can't use the \ARCode{ARController} API, as it hides this part of the protocol.}

\subsubsection{Routing video}

Routing an \ARCode{ARStream1} stream is done in the same way as routing the commands: All stream packets received from the drone are sent to the tablet (at the \ARCode{ARNetwork} level).

Routing an \ARCode{ARStream2} stream is done by using the \ARCode{libARStream2} library, and specifically the \ARCode{resender}-related functions. You can find informations about this API in the following file:\\
\ARFile{libARStream2}{Includes/libARStream2/arstream2_stream_receiver.h}


%%% Local Variables:
%%% mode: latex
%%% TeX-master: "main"
%%% End: